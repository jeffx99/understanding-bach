\section{Cadences}

A cadence is largely characterized by the final motion in its (implied) harmonic progression, as this terminates the phrase. Accordingly, we describe the common cadences first by the final bass motion and then by the harmonies over this bass motion. Typical cadences can be divided into two classes: one from a dominant harmony to a tonic harmony, and another from a non-dominant(predominant) harmony to a dominant harmony. The \say{strength} of a cadence refers to the degree of separation between adjacent phrases the cadence provides; this can be likened to the difference between a comma and semicolon or period. The strength of a cadence is related to its frequency of occurence---weak cadences are used infrequently, moderately strong ones are used most frequently, and especially strong ones are reserved for particular points in a piece.

\subsection{Dominant -- Tonic}
The first class of cadences, the \textbf{authentic cadences}, consist of motion from a dominant harmony to a tonic harmony. These form a typically strong group of cadences, as the motion to the final harmony either reaffirms the local key or introduces a new one. When a bass is present, the possible bass motions are(roughly in order of decreasing strength): \degree{5} -- \degree{1}, \, \degree{7} -- \degree{1}, \, \degree{2} -- \degree{1}, \, \degree{4} -- \degree{3}, \, \degree{5} -- \degree{3}, and \degree{2} -- \degree{3}.


The \degree{5} -- \degree{1} bass motion may be harmonized by giving \figured{5,3} or \figured{7,5,3} to \degree{5} and \figured{5,3} to \degree{1}. This type of harmonization is referred to as a \textbf{simple cadence}. Another possible harmonization gives \figured{6-5,4-3} to \degree{5}(with the seventh being optional as well); as two harmonies are given to a single bass note, this is referred to as a \textbf{compound cadence}. Since the harmonic rhythm is typically steady up until the final harmony(which may have an extended duration), the compound cadence usually has \degree{5} articulated for twice the duration than the other bass notes. Both of these cadences are given a stronger sense of finality by having the highest voice articulate \degree{1} by the end of the cadence; following proper voice-leading implies that it must also take the third or fifth of \degree{5}. When this occurs(which is usually at the end of a piece or section), the cadence is said to be \textbf{perfect}.
\begin{figure}[h]
\centering
\lilypondfile[staffsize=15]{chapters/phrases/excerpts/939.ly}
\vspace{-5pt}
\caption{A perfect compound cadence. \BWV{939}}
\end{figure}
\begin{figure}[h]
\centering
\lilypondfile[staffsize=15]{chapters/phrases/excerpts/880a.ly}
\caption{An imperfect simple cadence. By the time the suspensions resolve, the bass is no longer articulating \degree{1}, but the tonic harmony remains.  \BWV{880a}}
\end{figure}
The \degree{7} -- \degree{1} bass motion may be harmonized by giving \figured{5, 3}, \figured{6,3} or \figured{6,5,3}(yielding the strongest cadence) to \degree{7} and \figured{5,3} to \degree{1}. Almost invariably, the leading tone resolves in the expected fashion, moving up by step to \degree{1}. In cases where the bass appears to leap down a seventh, \degree{7} is almost invariably preceded by \degree{5} or \degree{2}(from underneath) on a stronger beat, suggesting a \degree{5} -- \degree{1} or \degree{2} -- \degree{1} cadence instead. The \degree{2} -- \degree{1} bass motion yields the weakest of the cadences ending on \degree{1}. One gives \figured{6,3} or \figured{6,4,3} to \degree{2} and \figured{5,3} to \degree{1}. Typically, this cadence is found when the bass is already engaged in a larger scalar motion down to \degree{1}.

\begin{figure}[h]
\centering
\begin{minipage}[b]{.45\textwidth}
\centering
\lilypondfile[staffsize=15]{chapters/phrases/excerpts/819.ly}
\vspace{6pt}
\end{minipage}
\begin{minipage}[b]{.45\textwidth}
\centering
\lilypondfile[staffsize=15]{chapters/phrases/excerpts/1010.ly}
\vspace{3pt}
\end{minipage}
\begin{minipage}[t]{.45\textwidth}
\caption{A \degree{7}--\degree{1} cadence. \BWV{819}}
\end{minipage}
\begin{minipage}[t]{.45\textwidth}
\caption{A \degree{5}--\degree{1} cadence containing a leap by seventh from \degree{7} to \degree{1}. \BWV{1010}}
\end{minipage}
\end{figure}

\begin{figure}[h]
\hspace{-8pt}
\centering
\lilypondfile[staffsize=13]{chapters/phrases/excerpts/813ii.ly}
\caption{A descending bass line starting from \degree{5} in \textbf{c} minor and descending by step, ending in a \degree{2} -- \degree{1} cadence in \textbf{f} minor. \BWV{813}}
\end{figure}
\begin{figure}[h]
\centering
\lilypondfile[staffsize=15]{chapters/phrases/excerpts/799.ly}
\caption{A ascending--descending bass line starting at \degree{1} and \say{peaking} at \degree{4}, ending in a \degree{2} -- \degree{1} cadence \BWV{799}}
\end{figure}


The \degree{4} -- \degree{3} bass motion is typically harmonized by giving \figured{6,4,2} to \degree{4} and \figured{6,3} to \degree{3}. This motion is the strongest of the cadences ending on \degree{3}, and is roughly comparable in strength to the \degree{2} -- \degree{1} cadence. When used, it is sometimes an example of \textbf{evasion}; this occurs when a potential \degree{5} -- \degree{1} cadence is forgone in favor of a \degree{5} -- \degree{4} -- \degree{3} cadence. The \degree{5} -- \degree{3} bass motion must be harmonized by giving \figured{5, 3} to \degree{5} and \figured{6, 3} to \degree{3}. Although this cadence places the dominant in the bass, it is weaker than the \degree{4} -- \degree{3} cadence, as it is impossible to have the seventh of \degree{5} in an upper voice(which would strengthen the cadence) without causing both direct octaves and doubling of the third, an unsatisfactory result. However, the impression of a \degree{4} -- \degree{3} cadence can be created by having the bass step through \degree{4} on a weaker beat; this is virtually always done by Bach.
\begin{figure}[h]
\centering
\lilypondfile[staffsize=15]{chapters/phrases/excerpts/870b.ly}
\caption{A \degree{5} -- \degree{4} -- \degree{3} cadence. The listener's expectations are not subverted by the \say{evasion} as the bass restates the fugue subject. \BWV{870b}}
\end{figure}
\begin{figure}[h]
\centering
\lilypondfile[staffsize=15]{chapters/phrases/excerpts/794.ly}
\caption{A \degree{5} -- \degree{3} cadence, in which \degree{4} is introduced as a passing tone. \BWV{794}}
\end{figure}
The \degree{2} -- \degree{3} bass motion yields the weakest of the authentic cadences for several reasons. One may give \figured{6,3}, \figured{6,4}, or \figured{6,4,3} to \degree{2} and \figured{6,3} to \degree{3}. If one gives the third to \degree{2}, it will either must rise by step(which is not its melodic tendency) or result in a doubling of the third. Furthermore, since the sixth of \degree{2} must resolve up by step and both chords share \degree{5}, the overall harmonic motion is parallel. Finally, neither \degree{2} or \degree{3} is the root of their respective harmony.

The reader may wonder why certain bass motions were omitted. After all, if four notes of the key can contribute to a dominant harmony and two to a tonic harmony then there should be 8 possible bass motions. However, the bass motions \degree{4} -- \degree{1} and \degree{7} -- \degree{3} do not obey the typical melodic tendencies of \degree{4} and \degree{7} to a tonic harmony. The latter bass motion does not exist as a cadence and the former is simply too uncommon to warrant discussion here.


\subsection{Predominant -- Dominant}

The second class of cadences consists of motion to a dominant harmony. These are referred to as \textbf{half cadences}, as they are generally weaker and set up an expectation for a stronger authentic cadence from dominant back to tonic. When a bass is present, the two most typical bass motions are \degree{4} -- \degree{5} or \degree{6} -- \degree{5}, with the former generally being more frequent.

The \degree{4} -- \degree{5} bass motion may harmonized in a multitude of ways. One gives \figured{5,3}, \figured{6,3}, or \figured{6,5,3} to \degree{4}, the latter usually yielding the most desirable cadence. In a major key, it is sometimes permissible to sharpen \degree{4} to strengthen the harmony, thereby creating the same resolution as a \degree{7} -- \degree{1} cadence in the dominant key. If the local key is minor, \degree{4} may be sharpened only if the third is as well; if so, one may use \figured{7, 5, \natural 3} in place of \figured{5,\natural 3}.

The \degree{6} -- \degree{5} bass motion is typically harmonized by giving \figured{6,3} or \figured{6,4,3} to \degree{6} and \figured{5,3} to \degree{5}. In a major key one typically improves the harmony by sharpening the sixth of \degree{6}, thereby creating the same resolution as a \degree{2} -- \degree{1} cadence in the dominant key. In a minor key, \degree{6} is occasionally sharpened; if so, then the sixth of $\sharp$\degree{6} may be sharpened as well.

\begin{figure}[h]
\centering
\lilypondfile[staffsize=15]{chapters/phrases/excerpts/811.ly}
\caption{A \degree{4} -- $\sharp$\degree{4} -- \degree{5} cadence, followed by a \degree{5} -- \degree{4} -- \degree{3} bass motion(\textbf{not} a cadence) into the next phrase. \BWV{811}}
\end{figure}


\subsection{Identifying Cadences}

A few characteristics of baroque music should be kept in mind when identifying cadences. It is commonplace for motoric rhythms to prevent the sense of any rhythmic \say{pause} at a cadence, as the music seamlessly transitions from one phrase to the next. This is to say that cadences in Bach's music are often \textbf{elided}, which is a weakening of the cadence by introducing the next phrase precisely at(or before) the cadence. Typically, a phrase is dominated by one or two motifs; when an elided cadence occurs a new motif is typically introduced during the cadence to be used during the next phrase. Another possible concern arises when the bass line is complex and touches on many essential notes, or is not present at all, making identification more difficult. In the first case, the cadence can not be neatly categorized as one the the previously mentioned types; however, the strongest impressions are made by the bass on the strong beats, which may be used to determine the overall impression of the cadence. Common examples of these ambiguities include a \degree{5} -- \degree{7} -- \degree{1} bass motion where \degree{7} is on a weaker beat than \degree{5}, or a \degree{6} -- \degree{4} -- \degree{5} bass motion where \degree{4} is on a weaker beat than \degree{6}. If the bass is absent, one may use the implied harmonic progression, or fit an implied bass line to determine the type and strength of a cadence. Finally, as general note, one should check that a tentative phrase does not render a following unit of music “incomplete”. Usually, this means that if a stronger cadence immediately proceeds a potential cadence, then one should consider the latter as closing off the entire phrase.
\begin{figure}[h]
\centering
\begin{minipage}[t]{.475\textwidth}
\centering
\lilypondfile[staffsize=15]{chapters/phrases/excerpts/789.ly}
\end{minipage}
\hfill
\begin{minipage}[t]{.475\textwidth}
\centering
\lilypondfile[staffsize=15]{chapters/phrases/excerpts/850a.ly}
\vspace{-16pt}
\end{minipage}
\begin{minipage}[t]{.4\textwidth}
\centering
\caption{A \degree{6}--\degree{5} cadence in a minor key. \BWV{789}}
\end{minipage}
\hspace{0.10\textwidth}
\begin{minipage}[t]{.4\textwidth}
\centering
\caption{A \degree{6}--\degree{5} cadence in a major key with a sharpened sixth of \degree{6}. \BWV{850a}}
\end{minipage}
\end{figure}


\begin{example}[\bwv{862a}]

To illustrate the deconstruction of a piece into its constituent phrases, consider the prelude in \textbf{f}$\sharp$ minor from the Well Tempered Clavier I. The prelude is written in \raisebox{-3pt}{\lilypond[inline, staffsize=13]{\markup{\column{\line{\compound-meter #'(4 4)}}}}} and consists of 24 measures. Table \ref{tbl:862a_phrases} shows each of the individual phrases and the cadences which demarcate them. One should also note that there is much more structure to this piece than the simple division into basic phrases. Often phrases are part of a hierarchical structure, with groups(usually pairs) of phrases forming a larger phrase.

Three perfect cadences are found in the piece: one in the middle confirming the minor dominant key, \textbf{c}$\sharp$ minor, and two at the end confirming the home key, \textbf{f}$\sharp$ minor. The two perfect cadences at the \say{coda} are an example of a typical pattern of \textbf{strengthening} or \textbf{prolongation} used to close a piece of music---one perfect cadence confirms the home key, and a stronger cadence in the home key follows shortly to conclude the piece. Generally, the first cadence is used to set up a tonic pedal point and/or is preceded by a dominant pedal point. The idea of strengthening a cadence may apply elsewhere in a piece to other types of cadences; one will note two other cadences as marked on Table \ref{} are strengthenings. In these particular cases, the material between the internal cadence and the end of the phrase can't stand alone as a phrase(one might say that it is an \textit{incomplete} subphrase).
\begin{table}[h]
\renewcommand{\arraystretch}{1.09}
\centering
\begin{tabular}{M|c|l}
\hline\hline
\multicolumn{2}{c|}{measures}	& cadence                   & remarks \\ \hline
1 & 2    & \cellcolor{gray!20} & authentic cadence; bass pedal point on \degree{1}	\\ \hline
2 & 3    &	\degree{4} -- \degree{3} 	& \\ \hline
3 & 5    &	\degree{4} -- \degree{3}	&	\\ \hline
5 & 7    &	\degree{2} -- \degree{1}	& compound bass; \degree{7} on weak beat following \degree{2} \\ \hline
7 & 8    & \degree{4} -- \degree{3}	&	\\ \hline
8 & 10		& \degree{2} -- \degree{1}	&	compound bass; \degree{7} on weak beat following \degree{2} \\ \hline
10 & 12	& \degree{5} -- \degree{1} &	preceded by deceptive \degree{5} -- \degree{6} cadence	\\ \hline
12 & 13  & \degree{2} -- \degree{1}	&	\\ \hline
13 & 14  & \degree{4} -- \degree{3}  & \\ \hline
14 & 15  & \degree{2} -- \degree{3}	&	\\ \hline
15 & 16  & \degree{2} -- \degree{1}  & \\ \hline
16 & 19	& \degree{6} -- \degree{5}	& strengthening of \degree{4} -- \degree{5} half cadence at mm.18	\\ \hline
19 & 20	& \degree{4} -- \degree{3}	&	upper voices absent on first beat of mm. 20, weakening cadence\\ \hline
20 & 22	& \degree{5} -- \degree{1}	&	begins a (thematic) pedal point on \degree{1} \\ \hline
22 & 23  & \degree{2} -- \degree{1}  & \\ \hline
23 & 24  & \degree{5} -- \degree{1}	& strengthening of \degree{4} -- \degree{3} cadence at mm. 24 \\ \hline\hline
\end{tabular}
\caption{The phrases and cadences of the \textbf{f}$\sharp$ minor prelude of WTC I. Due to the descending motif found throughout the piece, there are more \degree{2} -- \degree{1} cadences than typical. \BWV{862a}}
\label{tbl:862a_phrases}
\end{table}
\end{example}
